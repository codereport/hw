\documentclass{article}

% Formatting
\usepackage[utf8]{inputenc}
\usepackage[margin=1in]{geometry}
\usepackage[titletoc,title]{appendix}

% Math
% https://www.overleaf.com/learn/latex/Mathematical_expressions
% https://en.wikibooks.org/wiki/LaTeX/Mathematics
\usepackage{amsmath,amsfonts,amssymb,mathtools}

% Images
% https://www.overleaf.com/learn/latex/Inserting_Images
% https://en.wikibooks.org/wiki/LaTeX/Floats,_Figures_and_Captions
\usepackage{graphicx,float}

% Tables
% https://www.overleaf.com/learn/latex/Tables
% https://en.wikibooks.org/wiki/LaTeX/Tables

% Algorithms
% https://www.overleaf.com/learn/latex/algorithms
% https://en.wikibooks.org/wiki/LaTeX/Algorithms
\usepackage[ruled,vlined]{algorithm2e}
\usepackage{algorithmic}

% Code syntax highlighting
% https://www.overleaf.com/learn/latex/Code_Highlighting_with_minted
\usepackage{minted}
\usemintedstyle{borland}

% References
% https://www.overleaf.com/learn/latex/Bibliography_management_in_LaTeX
% https://en.wikibooks.org/wiki/LaTeX/Bibliography_Management
\usepackage{biblatex}
\addbibresource{references.bib}

% Title content
\title{CP8201 Homework 2}
\author{Conor Hoekstra}
\date{October 31, 2020}

\begin{document}

\maketitle

\section{}

\subsection{Overview}
The main idea needed to solve this problem in O(nm) time complexity is a dynamic programming technique called memoization. The general idea is for each point, we will \textit{walk} around the circumference by skipping to the end of each line and then finding the next non-overlapping line. At each point, we will memoize the number of lines included so far between [start, end] where \textit{start} was your starting point and \textit{end} is your current position. For future points/lines, redundant computation can then be avoided. When \textit{walking} the circumference and searching for the \texit{next line}, only consider lines that extend in the direction you are \textit{walking}.

\paragraph{}
A key thing to understand about this problem is when \textit{lines intersect}. A line that has no end points of any other lines in between the start point (A) and end point (B) of the line (note: there are two ways to travel between two points on a circle, always travel the shortest of the two distances when checking for \textit{end points in between}) does not intersect with any line. Lines that \textit{do} have end points in [A, B] intersect with all of the lines that start at one of those \textit{in-between points} and that end at an end point outside of [A, B].

\subsection{Pseudocode}

\begin{algorithm}
\begin{algorithmic}
    \STATE{HashMap$\langle$Int, TreeMap$\langle$Int, Int$\rangle\rangle$  memo \textit{// Int, Int, Int = Start Point, End Point, Line Count}}
    \STATE{answer = 0}
    \FOR{each $point$}
        \STATE{line = find shortest line starting at this $point$}
        \STATE{start = line.start}
        \STATE{end   = line.end}
        \STATE{count = 0}
        \WHILE{end $<$ start}
            \STATE{count += 1}
            \STATE{memo[start][end] = count}
            \STATE{nextStart = find next nearest point \textit{// (not equal to end)}}
            \IF{memo.contains(nextStart)}
                \STATE{memoizedCount = max(find largest less than $start$ in memo[$nextStart$], 0)}
                \STATE{count += memoizedCount}
                \STATE{\textbf{break}}
            \ENDIF
            \STATE{nextLine = find shortest line starting at $nextStart$}
            \STATE{end = nextLine.end}
        \ENDWHILE
        \STATE{answer = max(answer, count)}
    \ENDFOR
\end{algorithmic}
\caption{Pseudocode}
\label{alg:example}
\end{algorithm}

\section{}
\subsection{Overview}
The solution for this problem is almost identical to the solution for Question 1. The only adjustment we need to make is in our definition of \textit{overlapping}. For this problem, \textit{overlapping} does not include the end points. Once that adjustment is made, the rest of the algorithm is identical.

\subsection{Pseudocode}

For the pseudocode, all we need to change is \textbf{not equal to end} to \textbf{can be equal to end}.

\begin{algorithm}
\begin{algorithmic}
    \STATE{HashMap$\langle$Int, TreeMap$\langle$Int, Int$\rangle\rangle$  memo \textit{// Int, Int, Int = Start Point, End Point, Line Count}}
    \STATE{answer = 0}
    \FOR{each $point$}
        \STATE{line = find shortest line starting at this $point$}
        \STATE{start = line.start}
        \STATE{end   = line.end}
        \STATE{count = 0}
        \WHILE{end $<$ start}
            \STATE{count += 1}
            \STATE{memo[start][end] = count}
            \STATE{nextStart = find next nearest point \textit{// (can be equal to end)}}
            \IF{memo.contains(nextStart)}
                \STATE{memoizedCount = max(find largest less than $start$ in memo[$nextStart$], 0)}
                \STATE{count += memoizedCount}
                \STATE{\textbf{break}}
            \ENDIF
            \STATE{nextLine = find shortest line starting at $nextStart$}
            \STATE{end = nextLine.end}
        \ENDWHILE
        \STATE{answer = max(answer, count)}
    \ENDFOR
\end{algorithmic}
\caption{Pseudocode}
\label{alg:example}
\end{algorithm}

\newpage{}
\section{}

\subsection{Overview}
For the final question, we start from our solution in Question 1 and make only a small change. Instead of just memoizing and keeping track of the \textit{number of lines}, we need to memoize and keep track of a list of lines.

\subsection{Pseudocode}

\begin{algorithm}
\begin{algorithmic}
    \STATE{HashMap$\langle$Int, TreeMap$\langle$Int, List[Int]$\rangle\rangle$  memo \textit{// Int, Int, List[Int] = Start Point, End Point, Lines}}
    \STATE{answer = []}
    \FOR{each $point$}
        \STATE{line = find shortest line starting at this $point$}
        \STATE{start = line.start}
        \STATE{end   = line.end}
        \STATE{lines = []}
        \WHILE{end $<$ start}
            \STATE{lines.push(line)}
            \STATE{memo[start][end] = lines}
            \STATE{nextStart = find next nearest point \textit{// (not equal to end)}}
            \IF{memo.contains(nextStart)}
                \STATE{memoizedLines = max(find largest less than $start$ in memo[$nextStart$], 0)}
                \STATE{lines.append(memoizedLines)}
                \STATE{\textbf{break}}
            \ENDIF
            \STATE{line = find shortest line starting at $nextStart$}
            \STATE{end = nextLine.end}
        \ENDWHILE
        \IF{lines.size() $>$ answer.size()}
            \STATE{answer = lines}
        \ENDIF
    \ENDFOR
\end{algorithmic}
\caption{Pseudocode}
\label{alg:example}
\end{algorithm}

\end{document}
