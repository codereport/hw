
\documentclass{article}

% Formatting
\usepackage[utf8]{inputenc}
\usepackage[margin=1in]{geometry}
\usepackage[titletoc,title]{appendix}

% Math
% https://www.overleaf.com/learn/latex/Mathematical_expressions
% https://en.wikibooks.org/wiki/LaTeX/Mathematics
\usepackage{amsmath,amsfonts,amssymb,mathtools}

% Images
% https://www.overleaf.com/learn/latex/Inserting_Images
% https://en.wikibooks.org/wiki/LaTeX/Floats,_Figures_and_Captions
\usepackage{graphicx,float}

% Tables
% https://www.overleaf.com/learn/latex/Tables
% https://en.wikibooks.org/wiki/LaTeX/Tables

% Algorithms
% https://www.overleaf.com/learn/latex/algorithms
% https://en.wikibooks.org/wiki/LaTeX/Algorithms
\usepackage[ruled,vlined]{algorithm2e}
\usepackage{algorithmic}

% Code syntax highlighting
% https://www.overleaf.com/learn/latex/Code_Highlighting_with_minted
\usepackage{minted}
\usemintedstyle{borland}

% References
% https://www.overleaf.com/learn/latex/Bibliography_management_in_LaTeX
% https://en.wikibooks.org/wiki/LaTeX/Bibliography_Management
\usepackage{biblatex}
\addbibresource{references.bib}

\usepackage{hyperref}

\newlength\myindent
\setlength\myindent{2em}
\newcommand\bindent{%
  \begingroup
  \setlength{\itemindent}{\myindent}
  \addtolength{\algorithmicindent}{\myindent}
}
\newcommand\eindent{\endgroup}

% Title content
\title{CP8201 Homework 3 - Conor Hoekstra}
\author{Conor Hoekstra\\Reminder: you approved an extension for this assignment to Nov 16}
\date{November 15, 2020}

\begin{document}

\maketitle

\section{}

\subsection{}

The main idea here is to implement a function inverse that takes a polynomial \textit{f} and an integer \textit{t} and finds the 1/2\textsuperscript{t}-approximation of the inverse of \textit{f} by diving 1 by the polynomial \textit{f}. We can do this by representing our polynomial \textit{f} by an array of floating point numbers. We then perform long division of \textit{f} into 1.

\begin{algorithm}
\begin{algorithmic}
    \STATE{type polynomial = array}
    \STATE{\textbf{function} inverse(f: polynomial, t: int)}
        \bindent
        \STATE{result = polynomial of length t}
        \STATE{temp = polynomial of length t \textit{// use this for keep track of the remainder}}
        \STATE{temp[0] = 1 \textit{// initialize temp to be the polynomial representing a single 1}}
        \FOR{i = 0; i $<$ t; i += 1}
            \STATE{coefficient = temp[i] / f[0]}
            \STATE{result[i] = coefficient}
            \FOR{j = 0; j + i $<$ t; j += 1}
                \STATE{temp[j + i] -= coefficient * f[j]}
            \ENDFOR
        \ENDFOR
        \RETURN{result}
        \eindent
    \STATE{\textbf{end function}}
\end{algorithmic}
\caption{Pseudocode}
\label{alg:example}
\end{algorithm}

\subsection{}
The time complexity of my function is O(t\textsuperscript{2}) as we have two nested for loops with that are bound by \textit{t}.

\subsection{}
As a bonus, compiling / working solution in C++ \underline{\href{https://www.godbolt.org/z/bfccob}{can be seen here}}.

\end{document}
